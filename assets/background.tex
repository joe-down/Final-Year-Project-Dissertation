\section{Molecular Graphics}
The ability to visualise outputs from molecular simulations, particularly in the domain of liquid crystals, is important for understanding and communicating findings. QMGA\cite{gabriel2008molecular} is a tool which can be used to generate 3D graphical representations of molecular configurations. Despite being unmaintained since 2009\cite{qmga_release}, it remains in active use to this day, having been used within the last year in publications by notable authors such as \textcite{ramirez2023densest,mazzilli2023phase}. While another visualisation tool exists within the liquid crystal domain, LCview\cite{james2006finite,LCview}, it produces plots of director and/or potential fields, rather than showing the structure of large multi-molecule system.

Since QMGA has not been updated in so long, it continues to depend on the severely dated Qt 3 framework (Qt 4 was released in 2005, 2 years before QMGA was released) requiring the installation of unmaintained and difficult to acquire libraries (e.g. the Debian Linux distribution removed all Qt 3 libraries in 2012\cite{qt3_removed}). Additionally, since the program is distributed as source code, it must be manually built by the user which is not trivial for inexperienced users. This is complicated further by the fact that modern C compilers fail without certain modifications to the source code (described by \textcite{Battistini_2021} in their ``QMGA Compilation Issues Report''). All of these problems make installation on a modern system a significant barrier to usage. In preparation for this project, a successful install of QMGA was completed within a Debian 6 virtual machine, which verified the claims made regarding install difficulties.

WebMGA is a project begun by \textcite{Battistini_2021} in 2021 which aims to address this accessibility issue whilst replicating the functionality of QMGA. It was continued in 2023 by \textcite{webmga_2}. It's written in JavaScript using the React\cite{react} framework with the three.js\cite{three} library for 3D rendering. This addresses the accessibility issues of QMGA since it can be easily accessed using just a web browser. While WebMGA contains full functionality for rendering most molecule configurations from QMGA, it still has performance and functionality limitations, as well as some bugs. WebMGA 3.0 aims to address a majority of these issues.

\section{Liquid Crystal Modelling}
Most details regarding molecular simulation is not required to understand the implementations made for this project. Some key concepts which will be used throughout are defined in the following subsections, based on \textcite{allen2017computer} except where cited otherwise.
\subsection{Director}
\label{director_explain}
Under a coarse-grained potential model, liquid crystal configuration have a long-range orientational order, and sometimes also a long range positional order. An order represents a preferred alignment for molecules within that system\cite{dong1997orientational}. The orientational order can be described by a magnitude $S$, and a direction $\mathbf{n}$. $\mathbf{n}$ is typically referred to as the director. Both $S$ and $\mathbf{n}$ can be derived from the order tensor $\mathbf{Q}$, which is defined as follows for a system containing $N$ molecules each with unit vector principal axis direction $\mathbf{e}$,

\begin{equation}
\mathbf{Q}=
\frac{3}{2N}
\sum_{i=1}^{N}
(\mathbf{e}_i
\otimes
\mathbf{e}_i)
-\frac{1}{2}\mathbf{I}
\end{equation}

\begin{equation}
\mathbf{e}_i=\begin{pmatrix}
  \mathbf{e}_{ix}\\
  \mathbf{e}_{iy}\\
  \mathbf{e}_{iz}
\end{pmatrix}.
\label{order_tensor_e}
\end{equation}
The director $\mathbf{n}$ is defined as the eigenvector that corresponds to the maximum eigenvalue of the order tensor $\mathbf{Q}$.
%the direction which maximises $S$ and is obtainable by diagonalising $\mathcal{Q}$ and taking the eigenvector corresponding to the largest eigenvalue (which is itself $S$).

Since the director is a useful property for describing and understanding a system, it is convenient to be able to visualise both the director itself and how each molecule in the system aligns with it. Implementation of a director axis and director based axis colouring is discussed further in \cref{colour_director,director_axis_plot}, while molecule colouring based on director alignment was already implemented by \textcite{Battistini_2021}.

\subsection{Periodic Boundary Conditions}
\label{pbc_explain}
Typically molecular simulations can only be performed on small systems of $10<N<10,000$ molecules due to speed and/or storage constraints. With systems of this size, which is insufficient for simulating bulk liquids. Periodic boundary conditions allow modelling only a portion of an entire system\cite{gabriel2008molecular}, where an infinite lattice is simulated by repeating a smaller simulation box\cite{wu2014applying}. In each repeated boxes, periodic images of the molecules in the simulation box move in the exact same way. When a molecule leaves through a face of the simulation box, one of its periodic images enters through the opposite face of the simulation box with the same movement properties.

Due to the necessary use of periodic boundary conditions for many molecular simulations, it may be useful to visualise a larger portion of the infinite lattice, as discussed in \cref{peridodic_section}. Folding and unfolding of molecules based on periodic boundary conditions was already implemented by \textcite{webmga_2}.

\subsection{Molecule Shapes}
Molecular simulations model liquids as a collection of molecules. Each molecule has a shape approximately representative of the positions and sizes of the atoms they consist of. \textcite{allen2019molecular} identifies a range of appropriate molecule geometries commonly used for hard-particle simulations; specifically the hard sphere, prolate ellipsoid, spherocylinder, double cut sphere, and spheroplatelet. In addition to these, QMGA implements the eyelens shape. \textcite{cinacchi2019hard} have researched packings of molecules with a biconvex lens shape. All of these molecule shapes should be supported by WebMGA to successfully visualise models of each type. Implementations for geometry generation of all required types are discussed in \cref{shapes_section}.
