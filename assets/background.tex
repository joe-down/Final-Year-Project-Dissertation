\section{Molecular Graphics}
The ability to visualise outputs from molecular simulations, particularly in the domain of liquid crystals, is important for understanding and communicating findings. QMGA\cite{gabriel2008molecular} is a tool which can be used to generate 3D graphical representations of molecular configurations. Despite being unmaintained since 2009\cite{qmga_release}, it remains in active use to this day, having been used within the last year in publications by notable authors such as \textcite{ramirez2023densest,mazzilli2023phase}. While another visualisation tool exists within the liquid crystal domain, LCview\cite{james2006finite,LCview}, it produces plots of director and/or potential fields, rather than showing the structure of large multi-molecule system.

Since QMGA has not been updated in so long, it continues to depend on the severely dated Qt 3 framework (Qt 4 was released in 2005, 2 years before QMGA was released) requiring the installation of unmaintained and difficult to acquire libraries (e.g. the Debian Linux distribution removed all Qt 3 libraries in 2012\cite{qt3_removed}). Additionally, since the program is distributed as source code, it must be manually built by the user which is not trivial for inexperienced users. This is complicated further by the fact that modern C compilers will fail without certain modifications to the source code CITE THIS!!!. All of these problems make installation on a modern system a significant barrier to usage.

WebMGA is a project begun by \textcite{Battistini_2021} in 2021 which aims to address this accessibility issue whilst replicating the functionality of QMGA. It was continued in 2023 by CITE YUE. It's written in JavaScript using the React framework with the three.js library for 3D rendering. This addresses the accessibility issues of QMGA since it can be easily accessed using just a web browser. While WebMGA contains full functionality for rendering most molecule configurations from QMGA, it still has performance and functionality limitations, as well as some bugs. WebMGA 3.0 aims to address a majority of these issues.

\section{Liquid Crystal Modelling}
